%%%%%%%%%%%%%%%%%%%%%%%%%%%%%%%%%%%%%%%t%%%%%%%%%%%%%%%%%
% Latex Introduction
%
% 														               	       Written by Kim, Wiback,
% 																            2016.04.21. Ver. 1.1.
%           															            2016.05.01. Ver. 1.2.
%%%%%%%%%%%%%%%%%%%%%%%%%%%%%%%%%%%%%%%t%%%%%%%%%%%%%%%%%





%%%%%%%%%%%%%%%%%%%%%%% Document Settings %%%%%%%%%%%%%%%%%%%%%%%



%%%%%
% Class
%%%%%
% Other {options}: slides, minimal, ...
\documentclass[12pt, oneside]{article} 



%%%%%%%
% Packages
%%%%%%%

%%% Layout
\usepackage{geometry}                		
% Other {options}: letterpaper, a5paper, ...
\geometry{a4paper}                   		

%%% Mathematics
% Loading mathematical symbols
\usepackage{amssymb}				
% Enabling advanced mathematical equation control
\usepackage{amsmath} 
% Enabling cancelation of mathematical equations
\usepackage{cancel}

%%% Enabling hyper links
\usepackage{hyperref}
\hypersetup{
    colorlinks = true,
    % Links (except url) in blues
    linkcolor = blue, 
    % Files in magentas
    filecolor = magenta, 
    % urls in cyan
    urlcolor = cyan, 
} 
\urlstyle{same} 

%%% Use pdf, png, jpg, or eps (Encapsulated PostScript) with pdflatex; Use eps in DVI mode.
\usepackage{graphicx}				



%%%%%%%%
% If necessary, 
%%%%%%%%
%\geometry{landscape}                		% Activate for rotated page geometry
%\usepackage[parfill]{parskip}    		% Activate to begin paragraphs with an empty line rather than an indent



%%%%%%%%
% Links to TOC
%%%%%%%%
\usepackage{eso-pic}
\usepackage{ifthen}
% Activate links to TOC.
\newboolean{linktoc}
% Make it false to disable the links.
\setboolean{linktoc}{true}  

%%% Custom functions (change vertical locations to avoid overlap.)
% Vertical location -0.5 from top (more minus to go down)
\newcommand\AtPageUpperRight[1]{\AtPageUpperLeft{
\put(\LenToUnit{\paperwidth},\LenToUnit{-0.5\paperheight}){#1}}}
% Vertical location 0.5 from bottom (more plus to go up)
\newcommand\AtPageLowerRight[1]{\AtPageLowerLeft{
\put(\LenToUnit{\paperwidth},\LenToUnit{0.5\paperheight}){#1}}}

%%% Dispatching clickable objects that lead to TOC.
\ifthenelse{\boolean{linktoc}}
{\AddToShipoutPictureBG{
% Horizontal location -70 (more minus to go left)
\AtPageUpperRight{\put(-70,0){\hyperref[toc]{To Contents}}} 
% Horizontal location -70 (more minus to go left)
\AtPageLowerRight{\put(-70,0){\hyperref[toc]{To Contents}}}}}



%%%%%%%%
% Line spacing
%%%%%%%%
% Use 1.3 for one and half, 1.6 for double.
\linespread{1} 



%%%%
% Title
%%%%
% Making the title
\title{\LaTeXe{} Introduction \thanks{For anyone who is interested in Latex}} % \function{} blank for spacing
% Using \~{} to denote ~ sign
\author{Kim, Wiback \thanks{Efficiency\~{} \href{mailto:kwb425@icloud.com}{kwb425@icloud.com}}} 
\date{2016.04.20}





%%%%%%%%%%%%%%%%%%%%% Main Documenting Basics %%%%%%%%%%%%%%%%%%%%%



%%%%%%%
% Preamble
%%%%%%%
\begin{document}
% Drawing the title
\maketitle
% Table of contents
\tableofcontents \label{toc}



%%%%%%
% Abstract
%%%%%%
\begin{abstract} 
You can learn very basics of using \LaTeXe as well as some advanced mathematical applications.
\end{abstract}



%%%%%%%%%
% Section: basics
%%%%%%%%%
% \chapter{} -> \section{} -> \subsection{} -> \subsubsection{} -> \paragraph{} -> \subparagraph{}
\section{Basics} % \chapter{} only works with book or report.



%%%%%%%%%%%
% Subsection: pages
%%%%%%%%%%%
% New page
\newpage
\subsection{Pages}
New page



%%%%%%%%%%
% Subsection: lines
%%%%%%%%%%
\subsection{Lines} % \part{} is an independent section (has nothing to do with the ordering).
% Grouping characters
Going off the right edge of the paper $\to$ \mbox{These strings and numbers (21312903) will be grouped together at any cases.} % $\to$ denotes right arrow.
% New line
\newline
`New line' % Quotation is done by ` and '.
\newline
``New line'' % Quotation is done by ` and '.



%%%%%%%%%%%%%%
% Subsection: hyphenation
%%%%%%%%%%%%%%
\subsection{Hyphenation}
% \\ is same as \newline.
Single-hyphenation \\ % Single
Double--hyphenation \\ % Double
Triple---hyphenation \\ % Triple
Minus sign: $-1$ \\ % Minus sign



%%%%%%%%%%%%%%
% Subsection: Internet links
%%%%%%%%%%%%%%
\subsection{Internet links}
% Merits of using \url: 
% 1. automatic ~ (instead of \~{})
% 2. giving link
% 3. different font for WWW address
\url{https://github.com/kwb425/EMCS_Git_Latex_Introduction.git}



%%%%%%%%%%%%%%%%
% Subsection: end of sentences
%%%%%%%%%%%%%%%%
\subsection{End of sentences}
% End of a sentence is followed by automatic spacing for next sentence
% ! (end) spacing
% ? (end) spacing
% . (end) spacing
% lowercase (end) spacing
% \@ (forced end) spacing
Sentence end with CAPITALS\@. 
% vertical \vdots, center \cdots, row \ldots, diagonal \ddots
Next Sentence\ldots 



%%%%%%%%%%%%%%%%
% Subsection: cross-reference
%%%%%%%%%%%%%%%%
\subsection{Cross-reference}
% Labelling 
\label{sample_label_1}
% Referencing
\ref{sample_label_1} on page
% Extracting page number of the label
~\pageref{sample_label_1} 



%%%%%%%%%%%%%
% Subsection: Footnotes
%%%%%%%%%%%%%
% All footnotes must be protected in \section methods.
\subsection{Footnotes\protect\footnote{sample\_footnote\_1}} 
footnote\footnote{sample\_footnote\_2}



%%%%%%%%%%%%%%
% Subsection: emphasizing
%%%%%%%%%%%%%%
\subsection{Emphasizing}
Three methods are \textbf{bold}, \emph{italic}, and \underline{underline}.



%%%%%%%%%%%%%
% Subsection: alignment
%%%%%%%%%%%%%
\subsection{Alignment}
% Centering environment
\begin{center}
center alignment
\end{center}
% Flushleft environment
\begin{flushleft}
left alignment
\end{flushleft}
% Flushright environment
\begin{flushright}
right alignment
\end{flushright}



%%%%%%%%%%%%%%%%%%%%%%%%%%
% Subsection:  enumerates & items & descriptions
%%%%%%%%%%%%%%%%%%%%%%%%%%
\subsection{Enumerates \& Items \& Descriptions}

%%% Enumerate has orderings.
\begin{enumerate}
% All objects (enumerates & items & descriptions) will be called by \item method.
\item Sample\_enumerate\_1

%%% Item has no orderings.
\begin{itemize}
\item Sample\_item\_1
\item[-] Sample\_item\_2
\end{itemize}
% Second enumerate
\item Sample\_enumerate\_2

%%% Description has no orderings.
\begin{description}
\item [Sample\_description\_1] is a sample. % Bold with []
\end{description}
\end{enumerate}



%%%%%%%%%%%%%
% Subsection:  quatation
%%%%%%%%%%%%%
\subsection{Quatation}
Below is sample quatation.
\begin{quote}
This sample is written by Kim, Wiback
\end{quote}



%%%%%%%%%%%%%%
% Subsection:  raw printing
%%%%%%%%%%%%%%
\subsection{Raw printing}
% verbatim environment provides raw printing area that has nothing to do with Latex.
\begin{verbatim}
Raw printing /~!@#$%^&*()_+
\end{verbatim}



%%%%%%%%%%%
% Subsection:  tables
%%%%%%%%%%%
\subsection{Tables}
% \begin{tabular}[pos]{table spec}
% [pos]: t (top), b (bottom), c (center)
% {table spec}: l (left alignment & column separator), r (right), c (center), p{width} (justifying with changing rows)
\begin{tabular}[t]{| l @{ bound } c @{ bound } r |} % @{argument} replaces column separator with it's argument.
% Row insertion through \hline
\hline 
% & for next column, \\ for next row  
1st row               & 1st sample  & 1st sample                       \\
2nd row              & 2nd sample & 2nd sample                      \\
% Partial row insertion through \cline{startcolumn-endcolumn} 
\cline{2-3} 
3rd row               & 3rd sample & 3rd sample                        \\ 
\hline 
\hline
Pi                        & \multicolumn{2}{ c |}{Value \& Rounding} \\ % \multicolumn{size}{table spec}{text}
\hline
$\pi$                   & 3.1416        & 3                                         \\
$\pi^{\pi}$           & 36.46          & 36                                       \\
$(\pi^{\pi})^{\pi}$ & 80662.7      & 80663                                 \\ 
\hline
\end{tabular}

%%%%%%%%%%%
% Subsection:  floats
%%%%%%%%%%%
% Floats are containers for things in a document that cannot be broken over a page.
\subsection{Floats}
% \begin{figure}[placement specifier] 
% \begin{table}[placement specifier]
% placement specifier:
% h (here), t (top), b (bottom), p (page where floats are temporarily stacked), ! (force dispatch)
% !hbp == force dispatch (ignoring some display limits), first here, then bottom, then page for floats

%%% Activate for drawing
%\begin{figure}[!hbp]
%\makebox[\textwidth]{\framebox[5cm]{\rule{0pt}{5cm}}}
%\caption{Five by Five in Centimetres.}
%\end{figure}





%%%%%%%%%%%%%%%%%%%% Advanced Main Documenting %%%%%%%%%%%%%%%%%%%%



%%%%%%%%%
% Section: basics
%%%%%%%%%
% 1. Empty spaces, rows have no meanings in mathematical context. Use \,, \quad, or \qquad instead.
% 2. No equation can be separated with a empty row.
% 3. Normal text has to be inserted by \textrm{} (auto-resizing in \displaymath environment).
\newpage
\section{Mathematical equations}



%%%%%%%%%%%%%%%%
% Subsection: inline equations
%%%%%%%%%%%%%%%%
% Inline mathematic expressions: $ expressions $
\subsection{Inline equations}
% The binding with {}
$ a^x+y \neq a^{x+y}$ \\\\ % \neq (not equal) 
% Inline mathematic expressions: \begin{math} expressions \end{math}
\begin{math} \lim_{n \to \infty} \sum_{k=1}^n \frac{1}{k^2} = \frac{\pi^2}{6} \end{math} % _subscript, ^superscript



%%%%%%%%%%%%%%%%%%%%%
% Subsection: new paragraphic equations
%%%%%%%%%%%%%%%%%%%%%
\subsection{New paragraphic equations}
% New paragraphic mathematic expressions: \begin{displaymath} expressions \end{displaymath}
\begin{displaymath}
\lim_{n \rightarrow \infty}
\sum_{k=1}^n \frac{1}{k^2}
= \frac{\pi^2}{6}
\end{displaymath}



%%%%%%%%%%%%%%%%
% Subsection: cross-reference
%%%%%%%%%%%%%%%%
\subsection{Cross-reference}
% Not like displaymath (paragraphic) or math (inline) environment, 
% equation (paragraphic) environment provides orderings.
\begin{equation}
% Labelling
\label{sample_equation_1}
\epsilon > 0
\end{equation}
% Calling the Label
From (\ref{sample_equation_1}), \ldots



%%%%%%%%%%%%%%%%%%%%%%%%
% Subsection: texts in mathematical equations
%%%%%%%%%%%%%%%%%%%%%%%%
\subsection{Texts in mathematical equations}

%%% Only mathematic expressions
Without texts:
\begin{equation}
% Moderate (\,), quite (\quad), a lot (\qquad) spacings
\forall x \in \mathbf{R}: \qquad x^{2} \geq 0 % \geq (greater or equal)
\end{equation}

%%% Texts + Equations
With texts:
\begin{equation}
% Either one among \text{}, \mbox{}, \textrm{} is fine.
x^{2} \geq 0 \qquad \textrm{for all }x \in \mathbf{R} 
\end{equation}



%%%%%%%%%%%%
% Subsection: Greeks
%%%%%%%%%%%%
\subsection{Greeks}
% Latin-like alphabet will not be presented in \alphabet manner, but rather as just the alphabet itself.
Lowers: $\alpha, \beta, \gamma, \delta, \epsilon, \zeta, \eta, \theta, \iota, \kappa, \lambda, \mu, \nu,
\xi, o, \pi, \rho, \sigma, \tau, \upsilon, \phi, \chi, \psi, \omega$ \newline
Uppers: $A, B, \Gamma, \Delta, E, Z, H, \Theta, I, K, \Lambda, M, N, \Xi, O, \Pi, P, \Sigma, T, \Upsilon, \Phi, X, \Psi, \Omega$



%%%%%%%%%%
% Subsection: roots
%%%%%%%%%%
\subsection{Roots}
$\sqrt[3]{x}$ \qquad
$\sqrt[4]{x^{2}+\sqrt[2]{y}}$ \qquad 
$\sqrt[3]{2}$



%%%%%%%%%%%
% Subsection: braces
%%%%%%%%%%%
\subsection{Braces}
$\overline{x + y}$ \qquad
$\underline{x + y}$ \qquad
$\underbrace{x + y + \cdots + z}_{26}$ \qquad
$\overbrace{x + y + \cdots + z}^{26}$



%%%%%%%%%%%%
% Subsection: vectors
%%%%%%%%%%%%
\subsection{Vectors}
\begin{displaymath}
% Two ways of vectorizing
\vec{a} \quad \overrightarrow{AB}
\end{displaymath}



%%%%%%%%%%%%%%%%%%%%%%%%%%%%%%%%
% Subsection: derivatives & integrals & summations & products 
%%%%%%%%%%%%%%%%%%%%%%%%%%%%%%%%
\subsection{Derivatives \& Integrals \& Summations \& Products }
% Creating a anonymous command (similar to @function of MATLAB)
\newcommand{\ud}{\textrm{d}} % Now, \ud refers  to 'd'.
Four processes.

%%% The derivation	
Derivatives:
\begin{displaymath}
y = x^{2} \qquad y' = 2 \cdot x \qquad y'' = 2 \qquad % \cdot for one center dot, \cdots for three center dots
\frac{\partial f(x)}{\partial x} = \frac{\partial g(x)}{\partial y}
\end{displaymath}

%%% The integration
Integrals:
\begin{displaymath}
% Single
\int_{-\infty}^{\infty} \, \ud x \, \ud y 
% Triple
\iiint_{-\infty}^{\infty} \, \ud x \, \ud y
\end{displaymath}

%%% The summation
summations:
\begin{displaymath}
% Sub-stacking
\sum_{\substack{0<x<n \\ 1<y<m}} P(x,y) =
% Subarray at center
\sum_{\begin{subarray}{c} x \in I \\ 1<y<m \end{subarray}} Q(x,y) \qquad
\end{displaymath}

%%% The product
products:
\begin{displaymath}
\prod_\epsilon^\infty
\end{displaymath}



%%%%%%%%%%%%
% Subsection: functions
%%%%%%%%%%%%
% \arccos, \cos, \csc, \exp, \ker, \limsup, \min, \arcsin, \cosh, \deg, \gcd, 
% \lg, \ln, \Pr, \arctan, \cot, \det, \hom, \lim, \log, \sec, \arg, \coth, \dim, 
% \inf, \liminf, \max, \sin, \sinh, \sup, \tan, \tanh
\subsection{Functions}
Example among many functions (cos, sin, exp, ...):
\begin{displaymath}
\lim_{x \rightarrow 0} \frac{\sin x}{x} = 1
\end{displaymath}



%%%%%%%%%%%%
% Subsection: fractions
%%%%%%%%%%%%
\subsection{Fractions}
\begin{displaymath}
\frac{x^{2}}{k+1} \qquad x^{\frac{2}{k+1}} \qquad x^{1/2} % Sometimes, just using / is better.
\end{displaymath}



%%%%%%%%%%%%%
% Subsection: binomials
%%%%%%%%%%%%%
\subsection{Binomials}
% Binomials is binomials because we divide by 2 (not like permutation).
\begin{displaymath}
\binom{n}{k} \qquad \mathrm{C}_n^k
\end{displaymath}



%%%%%%%%%%%%
% Subsection: brackets
%%%%%%%%%%%%
\subsection{Brackets}
Two ways of dealing with the brackets

%%% Automatic sizings
Automatic sizings:
\begin{displaymath}
% []
[a, b, c] \equiv [a, b, c] \,
% The automatic sizing
\left[ \frac{1}{1-x^{2}} \right]^3 \qquad
% ()
(a, b, c) \equiv (a, b, c) \,
% The automatic sizing
\left( \frac{1}{1-x^{2}} \right)^3 \qquad
% {}
{a,b,c} \neq \{a,b,c\} \,
% The automatic sizing
\left\{ \frac{1}{1-x^{2}} \right\}^3 \qquad
\end{displaymath}

%%% Self sizings
Self sizings: \\
$\Big( (x+1) \cdot (x-1) \Big)^{2}$ \qquad
$\big( \Big( \bigg( \Bigg($ \qquad
$\big\} \Big\} \bigg\{ \Bigg\{$ \qquad



%%%%%%%%%%%
% Subsection: arrays
%%%%%%%%%%%
\subsection{Arrays}
Three forms.

%%% Full matrix
Full matrix form:
\begin{displaymath}
% Displaying mathematical X
\mathbf{X} =
% Start an array with 3 columns.
\left( \begin{array}{ccc} % \left( for (
x_{11} & x_{12} & \ldots \\
x_{21} & x_{22} & \ldots \\
\vdots  & \vdots & \ddots
% Close the array.
\end{array} \right) % \right) for )
\end{displaymath}

%%% Half matrix
Half matrix form:
\begin{displaymath}
% Start an array with 2 columns.
y = \left\{ \begin{array}{cc} % \left{ for {
 a     & \textrm{if $good$} \\
 b+c & \textrm{well...}       \\
 l      & \textrm{good for you?}
\end{array} \right. % \right. for empty spacing
\end{displaymath}

%%% Separated matrix
Separated form:
\begin{displaymath}
\left( \begin{array}{c | c}
1 & 2 \\
\hline
3 & 4 
\end{array} \right) 
\end{displaymath}



%%%%%%%%%%%%%
% Subsection: multilines
%%%%%%%%%%%%%
% In this environment, one can use \\ (line switching operator).
\subsection{Multilines}
% eqnarray* will eraze all equation numberings.
\begin{eqnarray} 
f(x) = \cos x  \\
f'(x) = -\sin x \\
% Giving some spaces to both ends of the '='
\int_{0}^{x} f(y)dy & = & \sin x 
\end{eqnarray}



%%%%%%%%%%%%%%%%%%%%%%%%%%%%%%
% Subsection: advanced alignment with amsmath package
%%%%%%%%%%%%%%%%%%%%%%%%%%%%%%
\subsection{Advanced alignment with amsmath package}
Total five environments.

%%% Equation
Equation environment:
% This is regular equation (to tag out equation numberings, use *.).
\begin{equation} 
e^{\pi i} - 1 = 0
\end{equation}

%%% Split (used in equation environment)
% This environment provides table-like structure for mathematical equations.
Split environment:
\begin{equation}
\begin{split}
A & = \frac{\pi r^2}{2} \\
   & = \frac{1}{2} \pi r^2
\end{split}
\end{equation}

%%% Multiline
% This environment will automatically align first to left, last to right, others to center.
Multiline environment:
\begin{multline*}
p(x) = 3x^6 + 19x^3y^3 \\ 
p(x) = 3x^6 + 19x^3y^3 \\
p(x) = 3x^6 + 19x^3y^3 \\
p(x) = 3x^6 + 19x^3y^3
\end{multline*}

%%% Gather
% This is simply a multiline environment with centred lines.
Gather environment:
\begin{gather*} 
2x - 5y = 8 \\ 
3x^2 + 9y = 1
\end{gather*}

%%% Align (highly useful)
% '&' is a vertical aligner.
Align environment:
\begin{align*}
x          &= y       &  w      &= z                 &  a  &= b + c \\ 
2x        &= -y      &  3w    &= \frac{1}{2}z &  a  &= b       \\
-4 + 5x &= 2 + y & w + 2 &= -1 + w        &  ab &= cb
\end{align*}



%%%%%%%%%%%%%%%%%%%
% Subsection: delicate space control
%%%%%%%%%%%%%%%%%%%
\subsection{Delicate space control}
\begin{displaymath}
% Put some hollow spaces (phantom) of length equal to that of xy before superscript.
\Gamma_{xy}^{\phantom{xy}z}
\qquad \textrm{vs} \qquad
% Without the phantom spaces
\Gamma_{xy}^{z}
\end{displaymath}



%%%%%%%%%%%%%%%%
% Subsection: bold in equations
%%%%%%%%%%%%%%%%
\subsection{Bold in equations}
\begin{displaymath}
\mu, M \qquad
\boldsymbol{\mu}, \boldsymbol{M}
\end{displaymath}



%%%%%%%%%%%%%%%%%%%%%%%
% Subsection: Encapsulated PostScript, .eps
%%%%%%%%%%%%%%%%%%%%%%%
\subsection{Encapsulated PostScript, .eps}
% Figure will be drawn in a float.
\begin{figure}
\centering
\caption{Upper caption}
% Rotation of 90, horizontal size is half of \textwidth, and file is located at eps_files/test_image.eps.
\includegraphics[angle = 90, width = 0.5\textwidth]{eps_files/sample_image_1.eps}
\caption{Bottom caption}
\end{figure}
% Giving some spaces for the figure to be drawn.
\newpage



%%%%%%%%%%%%%
% Subsection: references
%%%%%%%%%%%%%
\subsection{References}
Kim \cite{sample_cite_1} has proposed that \ldots
% thebibliography environment 
\begin{thebibliography}{99} % Maximum number of references is, in this case, 99.
% Use \bibitem for each reference.
\bibitem{sample_cite_1} Kim:
\emph{Korea},
Introduction to \LaTeXe (2016)
\end{thebibliography}





%%%%%%%%%%%%%%%%%%%%%% End of the Document %%%%%%%%%%%%%%%%%%%%%%
\end{document}  